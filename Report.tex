\documentclass[a4paper]{article}

% Packages
\usepackage[utf8]{inputenc}
\usepackage{graphicx}
\usepackage{hyperref}
\usepackage{amsmath}
\usepackage{amsfonts}
\usepackage{amssymb}
\usepackage[left=2cm,right=2cm,top=2cm,bottom=2cm]{geometry}
\usepackage{fancyhdr}
\usepackage{glossaries}
\usepackage{listings}
\usepackage{xcolor} % Extended color functionalities

\hypersetup{colorlinks=true, linkcolor=blue}

% Define custom colors
\definecolor{codegreen}{rgb}{0,0.6,0}
\definecolor{codegray}{rgb}{0.5,0.5,0.5}
\definecolor{codepurple}{rgb}{0.58,0,0.82}
\definecolor{backcolour}{rgb}{0.95,0.95,0.92}

% MATLAB style for highlighting
\lstdefinestyle{mystyle}{
    backgroundcolor=\color{backcolour},   
    commentstyle=\color{codegreen},
    keywordstyle=\color{magenta},
    numberstyle=\tiny\color{codegray},
    stringstyle=\color{codepurple},
    basicstyle=\ttfamily\footnotesize,
    breakatwhitespace=false,         
    breaklines=true,                 
    captionpos=b,                    
    keepspaces=true,                 
    numbers=left,                    
    numbersep=5pt,                  
    showspaces=false,                
    showstringspaces=false,
    showtabs=false,                  
    tabsize=2
}

\lstset{style=mystyle}

% Glossary entries
\newacronym{pep}{PEP}{Peak Envelope Power}
\newacronym{papr}{PAPR}{Peak to Average Power Ratio}
\newacronym{ofdm}{OFDM}{Orthogonal Frequency Division Multiplexing}

% Document
\begin{document}

\begin{titlepage}
    \begin{center}
        \vspace*{1cm}

        \Huge
        \textbf{Modulation Waveforms Lab Report}

        \vspace{0.5cm}
        \LARGE
        EXPERIMENT CP-SRP
        EE20017
        
        \vspace{1.5cm}

        \textbf{Jake Stewart}\\
        \textbf{JS3910}\\

        \textbf{Eugene Levinson}\\
        \textbf{EL769}\\
        \vspace{0.8cm}

        \vfill
        \includegraphics[width=0.4\textwidth]{university_logo.png}

        \Large
        Electrical and Electronic Engineering\\
        University of Bath\\
        United Kingdom\\
        \today
        

    \end{center}
\end{titlepage}

\newpage
\tableofcontents
\newpage

\section{Introduction}
Introduce the topic of your report and provide background information.

\section{Modulation Tests}
A square wave and triangle wave were generated and modulated using an AM modulator.
The modulated waveforms were then observed in MATLAB where the \gls{pep} and \gls{papr} were derived with the following code:

\begin{lstlisting}[language=Matlab]
    % AM Power
    R = 50; % Ohms
    AM_Power = (AM_time.^2) / (2*R); % RMS Power in Watts V^2/2R
    % Peak Envelope Power
    PEP = max(AM_Power)
    % Peak to Average Power Ratio
    PAPR = PEP/mean(AM_Power)
\end{lstlisting}

\subsection{Results}
\subsection*{Square Wave}
\begin{itemize}
    \item \gls{pep}: 134.5mW
    \item \gls{papr}: 4.74
\end{itemize}

\subsection*{Triangular Wave}
\begin{itemize}
    \item \gls{pep}: 369.6mW
    \item \gls{papr}: 8.14
\end{itemize}

\section{Demodulation/detection}

\subsection{AM detection}

\subsection*{Phase offset in receiver carrier}

\subsection*{Frequency offset in receiver carrier}

\section{Orthogonal Frequency Division Multiplexing (OFDM)}
\subsection{OFDM Baseband Signal}
\subsection*{"S" Symbol @ 1V}
\begin{itemize}
    \item \gls{pep}: 287.8mW
    \item \gls{papr}: 8.22
\end{itemize}

\subsection*{"S" Symbol @ 2.5V}
\begin{itemize}
    \item \gls{pep}: 1.8W
    \item \gls{papr}: 8.22
\end{itemize}

\subsection{64-QAM on OFDM}

\subsection*{}
\begin{itemize}
    \item \gls{pep}: 1003W
    \item \gls{papr}: 6.64
\end{itemize}

\end{document}
