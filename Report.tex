\documentclass[15pt,a4paper]{article}

% Packages
\usepackage[utf8]{inputenc}
\usepackage{graphicx}
\usepackage{hyperref}
\usepackage{amsmath}
\usepackage{amsfonts}
\usepackage{amssymb}
\usepackage[left=2cm,right=2cm,top=2cm,bottom=2cm]{geometry}
\usepackage{fancyhdr}
\pagestyle{fancy}
\fancyhf{}
\rhead{Jake Stewart and Eugene Levinson}
\lhead{Modulation Waveforms Lab Report EE20017}
\rfoot{Page \thepage}

% Document
\begin{document}

\begin{titlepage}
    \begin{center}
        \vspace*{1cm}

        \Huge
        \textbf{Modulation Waveforms Lab Report}

        \vspace{0.5cm}
        \LARGE
        EXPERIMENT CP-SRP
        EE20017
        
        \vspace{1.5cm}

        \textbf{Jake Stewart}\\
        \textbf{JS3910}\\

        \textbf{Eugene Levinson}\\
        \textbf{EL769}\\

        \vfill
        
        A report presented for\\
        Communication Principles EE20017\\

        \vspace{0.8cm}

        %\includegraphics[width=0.4\textwidth]{university_logo.png}
        
        \Large
        Electrical and Electronic Engineering\\
        University of Bath\\
        United Kingdom\\
        \today

    \end{center}
\end{titlepage}

\newpage
\tableofcontents
\newpage

\section{Introduction}
Introduce the topic of your report and provide background information.
\section{Matlab Code}

\begin{verbatim}
    R = 50; % ohms
    AM_Power = AM_time.^2/R;
    PEP = max(AM_Power)/2 % divide by 2 because of root mean square
    PAPR = PEP/mean(AM_Power)
\end{verbatim}

\section{Theory}
Discuss the theoretical background of the experiments you conducted.

\section{Methodology}
Explain the methods used in your experiments.

\section{Results}
Present the results of your experiments.

\section{Discussion}
Discuss the significance of your results.

\section{Conclusion}
Conclude the report by summarizing the findings and their implications.


\end{document}
